
%----------------------------------------------------------------------------------------
%	SECTION 2
%----------------------------------------------------------------------------------------
\newpage
\section{Overview of the Candidate Architecture}
The system used in this laboration is implemented in either two configurations; centralized and distributed architectures. There are also two modes of operation, namely full and degraded functionality. Assumptions and modeling parameters for the two architectures described below apply to their respective configurations.

\subsection{Centralized Architecture}
The centralized architecture relies on the Central Unit (CU) to execute the control algorithm. The wheel units only contain sensors and actuators communicating with the CU over the communication network. With this approach the wheel units contain less hardware and become less complex, leading to lower failure rates in the wheel units. However, the central unit is more complex in this architecture since it requires more processing power and memory, which results in a higher failure rate of the central unit.\\
\\
\subsection{Distributed Architecture}
The distributed architecture relies on executing the control algorithm locally for each wheel unit. In this architecture the wheel units are more complex than in the centralized architecture approach. The wheel units are similar to the central unit in terms of complexity but suffers from a higher failure rate than the central unit since they are more vulnerable to vibrations, moisture and temperature cycling.
In this section you describe the two modes of operation of the system; full functionality and degraded functionality.
\subsubsection{Full Functionality}
The first mode of operation is full functionality. In order to provide full functionality, the central unit, all four wheel units and at least one system bus must be working.
\subsubsection{Degraded Functionality}
The second mode of operation is degraded functionality. In this mode, the system can no longer issue the brake command for one of the wheels. Because of this, the system is only 
operational if three wheel units, the central unit and at least one system bus are working. For a more simple analysis, the system is considered to have failed catastrophicly
if the system is unable to send its brake command to two or more wheels.
\subsection{Assumptions and modeling parameters}
/text/
\begin{table}[h]
\centering
\begin{tabular}{| c | c | c | c |}
\hline 
Subsystem & Part & Failure rate & Coverage\\
\hline
System bus & Serial bus& FailureRate & 1\\
\hline
Wheel unit & Computer module & FailureRate & 1\\
\hline
Wheel unit & Sensor & FailureRate & 1\\
\hline
Wheel unit & Actuator & FailureRate & 1\\
\hline
Central unit & Computer module & FailureRate & 0.99\\
\hline
\end{tabular}
\caption{Failure rates and coverage factors for the distributed architecture}
\label{tab:Put a Lable}
\end{table}
\begin{table}[h]
\centering
\begin{tabular}{| c | c | c | c |}
\hline 
Subsystem & Part & Failure rate & Coverage\\
\hline
System bus & Serial bus& FailureRate & 1\\
\hline
Wheel unit & Computer module & FailureRate & 1\\
\hline
Wheel unit & Sensor & FailureRate & 1\\
\hline
Wheel unit & Actuator & FailureRate & 1\\
\hline
Central unit & Computer module & FailureRate & First CM failure:1 Second CM failure: 0.99\\
\hline
\end{tabular}
\caption{Failure rates and coverage factors for the Centralized Architecture}
\label{tab:Put a Lable}
\end{table}
